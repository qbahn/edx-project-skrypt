\section{Przekroje}
Niech $A$, $B$ będą zbiorami liczb. Niech $a,b,p,q$ będą liczbami, elementami zbiorów.

Równość zbiorów $A$ i $B$ definiujemy następująco
$$ A = B \iff (a\in A \implies a\in B) \wedge (a\in B \implies a\in A), $$
mówimy, że dwa zbiory nie są równe, gdy
$$ A \neq B \iff \exists a \in A | a\notin B \vee \exists a \in B | a \notin A$$

Niech $A$ będzie zbiorem liczb wymiernych mniejszych niż $\sqrt{2}$, zauważmy, że wówczas
\begin{align*}
A & = \{ p \in \mathbb{Q} | p < \sqrt{2} \} \; \mbox{nie ma liczby największej} \\
B & = \{ p \in \mathbb{Q} | p > \sqrt{2} \} \; \mbox{nie ma liczby najmniejszej}.
\end{align*}

\begin{proof}
Pokażemy najpierw, że nie może w $A$ istnieć element największy. 
Niech $p\in A$, pokażemy, że istnieje $k>0$, że $(p+k)\in A$.

Zauważmy, że biorąc $k <\min \left( 1,\frac{2-p^2}{2p+1} \right)$ otrzymujemy
\begin{align*}
(p+k)^2 &< 2 \\
p^2 + 2pk + k^2 & < 2 \\
p^2 + k(2p+k) & < p^2 + \frac{2-p^2}{2p+1} \left( 2p + 1\right)  = p^2 . 
\end{align*}

Pokażemy teraz, że dla $q\in B$ znajdziemy takie $h >0 $, że $q-h \in B$.
Weźmy $h = \frac{q^2 - 2}{2q}.$  Łatwo możemy zauważyć, że $h \in \mathbb{Q}$ (jako iloraz dwóch liczb wymiernych).

Policzmy wartość 
$$ (q-h)^2 = q^2 - 2 \frac{q^2-2}{2q}q + h^2 = 2+h^2 > 2.$$
Tak więc dobierając $h$ w zadany sposób możemy zawsze wskazać liczbę mniejszą $q-h$, która także należy do $B$.
\end{proof}

\subsection{Gęstość liczb wymiernych}
Pomiędzy dwiema dowolnymi liczbami wymiernymi zawsze znajdziemy liczbę wymierną.

\begin{proof}
Niech $p,q\in \mathbb{Q}$ wówczas 
$$ p < \frac{p+q}{2} < q $$
oraz $$ \frac{p+q}{2} \in \mathbb{Q}.$$
\end{proof}

\subsection{Metoda przekrojów Dedekinda}
\begin{definicja}
Zbiór $\alpha$ liczb wymiernych nazywamy przekrojem, jeżeli:
\begin{enumerate}
\item[a)] $\alpha$ nie zawiera wszystkich liczb wymiernych oraz $\alpha$ zawiera co najmniej 1 liczbę wymierną,
\item[b)] $p\in\alpha, q<p \implies q\in\alpha$,
\item[c)] $\alpha$ nie ma liczby największej.
\end{enumerate}
\end{definicja}

Przekrojem jest na przykład zbiór $A$ z poprzedniego przykładu.

\begin{twierdzenie}
Jeżeli $p\in\alpha \wedge q\notin \alpha$, to $p<q$ dla $q\in\mathbb{q}$.
\end{twierdzenie}
\begin{proof}
Przypuśćmy nie wprost, że $q < p$, na mocy warunku $b)$ z definicjji przekroju wnioskujemy, że $q\in\alpha$, co jest sprzeczne z założeniem.
\end{proof}

\begin{twierdzenie}
Niech $r\in\mathbb{Q}$ oraz $\alpha = \{ p\in\mathbb{Q} | \, p<r \}$ - wtedy $\alpha$ jest przekrojem.
\end{twierdzenie}
\begin{proof}
Warunki $a)$, $b)$ są spełnione w sposób oczywisty.

Warunek $c)$ - niech $p\in\alpha$ czyli $p<r$. Wtedy $p < \frac{p+r}{2} < r$ oraz $\frac{p+r}{2}\in\alpha.$
\end{proof}

\begin{definicja}
Przekrój zadany jak w powyższym twierdzenimu nazywamy przekrojem wymiernym.
\end{definicja}
Zauważmy, że zbiór $\mathbb{Q}/\alpha$ ma liczbę najmniejszą $r$.

\begin{definicja}
Niech $\alpha$, $\beta$ będą przekrojami.
\begin{align*}
\alpha = \beta & \iff p\in\alpha \iff p\in\beta \\
\alpha < \beta & \iff \exists p \in B \,: \, p \notin \alpha \\
\alpha > \beta  & \iff \exists q \in \alpha \,:\, q \notin B \\
\alpha \leq \beta & \iff \alpha < \beta \vee \alpha = \beta \\
0^* & = \{ p\in\mathbb{Q} \, : \, p<0 \}
\end{align*}
 Mówimy, że 
\begin{enumerate}
\item[] $\alpha$ jest dodatni, jeżeli $\alpha > 0^*$,
\item[] $\alpha$ jest ujemny, jeżeli $\alpha < 0^*$,
\item[] $\alpha$ jest niejemny, jeżeli $\alpha \geq 0^*$,
\item[] $\alpha$ jest niedodatni, jeżeli $\alpha \leq 0^*$.
\end{enumerate}
\end{definicja}

\begin{twierdzenie}
Niech $\alpha$, $\beta$ będą przekrojami. Wtedy
$$ \alpha = \beta \quad \mbox{albo}\quad \alpha<\beta\quad\mbox{albo}\quad\alpha>\beta.$$
\end{twierdzenie}
	
\begin{proof}
Rozważmy dwa przypadki. Jeżeli każdy element $\alpha$ jest elementem $\beta$ i na odwrót to w oczywisty sposób przekroje te muszą być równe. 

W przeciwnym przypadku istnieje element, który nie należy do jednego z przekrojów, pokażemy, że nie może wówcas równocześnie zachodzić, że $\alpha >\beta \land \beta>\alpha.$


Dowód prowadzimy nie wprost

Niech $\alpha$ będzie zbiorem liczb mniejszych od $p$, a $\beta$ mniejszych od $q$. 
Wówczas\\
(1) $\alpha >\beta \Rightarrow \exists{p'\in \alpha}\,:\,p'\not \in \beta$\\
(2) $\beta >\alpha \Rightarrow \exists{q'\in \beta}\,:\,q'\not \in \alpha$\\
(1)$\implies p'\in \beta \Rightarrow p'<q \land q'\not \in \beta \implies q'>q \implies p'<q'$\\
(2)$\implies q'\in \alpha \implies q'<p \land p'\not \in \alpha \implies p'>p \implies q'<p'$.

 Stąd otrzymujemy, że $p'<q'<p'\implies p'<p'$ co jest sprzeczne.
\end{proof}

























