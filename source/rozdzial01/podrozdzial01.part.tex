\section{Liczby naturalne, całkowite i wymierne}
Zbiór liczb naturalnych oznaczamy jako $\mathbb{N}$ i jego definicja jest jak następuje
$$ \mathbb{N} := \{ 1, 2, 3, \ldots\}.$$

Liczby całkowite oznaczamy jako $\mathbb{Z}$ i definiujemy jako sume liczb naturalnych, liczb przeciwnych do naturalnych oraz zera
$$ \mathbb{Z} := \mathbb{N} \cup \{ a | -a \in \mathbb{N} \} \cup \{0\}. $$

Liczby wymierne oznaczamy jako $\mathbb{Q}$  i definiujemy następująco
$$ p \in \mathbb{Q} \iff p = \frac{n}{m} \quad \mbox{gdzie}\quad n,m \in \mathbb{Z}, m\neq 0.$$


\subsection{Własności liczb wymiernych}
Poniżej wypiszemy podstawowe własności liczb wymiernych
\begin{enumerate}
\item prawo przemienności: $\forall p,q\in\mathbb{Q}$
$$ p + q = q + p, \quad p \cdot q = q \cdot p$$
\item prawo łączności: $\forall p,q,r\in\mathbb{Q}$
\begin{align*}
(p+q) + r & = p + (q + r) \\
(p\cdot q) \cdot r & = q \cdot (q \cdot r)
\end{align*}
\item prawo rozdzielności dodawania względem mnożenia: $\forall p,q,r\in\mathbb{Q}$
$$ (p+q) \cdot r = p\cdot r + q\cdot r $$
\item relacja porządku $\forall p,q\in\mathbb{Q}$
$$ p=q \vee p<q \vee p>q $$
\item węwnętrzność działania dodawania i mnożenia dla liczb dodatnich
Niech $p>0 \wedge q>0$, wówczas
\begin{align*}
p+q & > 0 \\
p \cdot q & > 0
\end{align*}
\item warunek równości dwóch liczb wymiernych
Niech $a,b\in\mathbb{Q}$ oraz niech $a = \frac{p}{q}$ oraz $b = \frac{p'}{q'}$, wówczas
$$ a = b \iff p\cdot q' = p' \cdot q$$
Ponadto, zakładając $q, q' >0$ otrzymujemy
\begin{align*}
a<b & \iff p\cdot q' < p' \cdot q \\
a>b & \iff p\cdot q' > p' \cdot q 
\end{align*}
\end{enumerate}

\subsection{Niewymierność $\sqrt{2}$}

Pokażemy, że pierwiastek z dwóch nie jest liczbą wymierną, tj. $\sqrt{2} \notin \mathbb{Q}$.
%\begin{cor}
%Pierwiastek z dwóch jest liczbą niewymierną.
%\end{cor}
\begin{proof}
Dowód będziemy prowadzić nie wprost. 
Przypuśćmy nie wprost, że $\sqrt{2} \in \mathbb{Q}$, wówczas muszą istnieć $p,q\in\mathbb{N}$, że
$$ \sqrt{2} = \frac{p}{q},$$
a ułamek ten jest nieskracalny.

Równoważnie możemy napisać
\begin{align*}
q\sqrt{2} &= p \quad \mbox{podnosząc do kwadratu stronami}\\
2 q^2 &= p^2
\end{align*}
Zauważmy, że wynika stąd, że $p^2$ jest parzyste, to oznacza, że także $p$ jest parzyste, czyli $\exists n\in\mathbb{N}$, że $p = 2n$. Wstawiając, otrzymujemy
\begin{align*}
2q^2 &= 4 n^2 \\
q^2 & = 2 n^2
\end{align*}
Stosując powyższe rozumowanie otrzymujemy, że $\exists m \in \mathbb{N}$, że $q = 2m$. Widzimy więc sprzeczność z założeniem, że ułamek $\frac{p}{q}$ był nieskracalny (jako, że zarówno licznik jak i mianownik są podzielne przez 2). Sprzeczność ta dowodzi tezy.
\end{proof}
 
\subsection{Niewymierność $\log_{10} 5$}
Pokażemy, że liczba $\log_{10} 5$ jest niewymierna.
Podobnie jak powyżej, dowód będzie prowadzony nie wprost.

\begin{proof}
Przypuśćmy, że $\log_{10}5 = \frac{p}{q}$, gdzie $p,q\in\mathbb{N}$. Jest to równoważne następującemu wyrażeniu
\begin{align*}
 5 &= 10^{\frac{p}{q}} \\
 5^q & = 10^p.
\end{align*}

Zauważmy, że powyższa równość jest niemożliwa pośród liczb naturalnych (ze względu na podzielność przez $2$). 

\end{proof}


